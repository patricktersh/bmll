\documentclass[12pt]{article} 
% \documentclass[page-classic]{epl2} %for one column style

\usepackage{amsmath}
\usepackage{psfrag}
\usepackage[left=2cm,right=2cm,bottom=4cm, top=3cm]{geometry}
\newcommand{\df}[1]{\delta\!\left(#1\right)\!}
\newcommand{\ud}{ \mathrm{d}x\ }
\usepackage{amsmath}
\usepackage{amsfonts}
\usepackage{mathtools}
\usepackage{color}
\usepackage[pdftex]{graphicx}
\usepackage{courier}


\title{Price Impact Analysis}
\author{Elia Zarinelli}

\begin{document}

\maketitle


%\begin{abstract}
%This document details the analysis of price impact of large orders (metaorders) executed 

%We make an extensive empirical study of the market impact of large orders (metaorders) executed in the U.S. equity market between 2007 and 2009. We show that the square root market impact formula, which is widely used in the industry and supported by previous published research, provides a good fit only across about two orders of magnitude in order size.    A logarithmic functional form fits the data better, providing a good fit across almost five orders of magnitude.  We introduce the concept of an ``impact surface" to model the impact as a function of both the duration and the participation rate of the metaorder, finding again a logarithmic dependence. We show that during the execution the price trajectory deviates from the market impact, a clear indication of non-VWAP executions.  Surprisingly, we find that sometimes the price starts reverting well before the end of the execution. Finally we show that, although on average the impact relaxes to approximately $2/3$ of the peak impact, the precise asymptotic value of the price depends on the participation rate and on the duration of the metaorder. We present evidence that this might be due to a herding phenomenon among metaorders.


%\end{abstract}

This document provides an overview of the code \textsf{sources/imp\_tmp.py} that perform an analysis of the price impact of large orders executed on the U.S. equity market. This analysis is performed on a dataset containing 113.562 metaorders executed in 2007 on the equity GE, C, CSCO, PG, XOM, MSFT, BAC, JPM, AIG and MRK. This dataset is a subset of a larger dataset provided by ANcerno. The presentation of the analysis of the whole dataset can be found at the following address: http://arxiv.org/abs/1412.2152.

\section{Definitions}

We introduce three parameters characterising the execution of a metaorder buying/selling ($\epsilon = \pm 1$) $Q$ shares in a time interval $[t_s,t_e]$:
\begin{itemize}
\item  The \textbf{participation rate} $\eta$ is defined as the ratio between $Q$ and $V_P$, the volume traded by the whole market during the execution interval: $\eta := \frac{Q}{V_P}$.
\item The \textbf{duration} $D$ is defined as the ratio between $V_P$ and $V_D$, the volume traded by the market in the whole trading day: $D := \frac{V_P}{V_D}$.
\item The \textbf{daily fraction} $\pi$ is defined as the ratio between $Q$ and $V_D$:  $\phi := {Q}/{V_D}$. 
\end{itemize}
In order to define market impact, we define $s(t)$ as the logarithm of the price $S(t)$ at time $t$ rescaled by the daily volatility $\sigma_D$, i.e. $s(v) := {\log S(v)}/{\sigma_D}$. We consider the \textbf{temporary} market impact measured at the moment $t=t_e$ when the metaorder is completed as a function of the daily fraction, i.e.
\begin{equation} \label{eq_imp_tmp}
\mathcal{I}_{tmp}(\pi) := \mathbb{E}\left[ \left. \epsilon \left( s(t_e) -s(t_s) \right)  \right|  \pi \right].
\end{equation} 
and the \textbf{transient} market impact quantifies how market impact builds up during the execution of the metaorder,  i.e. $t_s<t< t_e$.
\\

\section{Datasets}
We consider two datests:
\begin{itemize}
\item The private dataset is a subset of the ANcerno database, containing 113.562 metaorders executed in 2007 on the equity GE, C, CSCO, PG, XOM, MSFT, BAC, JPM, AIG and MRK. This can be found in \textsf{data\_frame/df\_07\_sel.csv}. This dataset is the result of a data-mining operation on the raw dataset provided by ANcerno. For each metaorder, we are able to recover the following infromation: \textsf{symbol, side, Q, V\_P, V\_D, sigma\_D, t\_s, t\_e, p\_s, p\_e}. Within our code, we import the dataset in \textsf{df\_in}, a \textsf{DataFrame} object, and we calculate for each metaorder the quantities \textsf{pi, eta, dur, imp}, as defined before.

\item The public dataset contains the time series of the opening price recorded with a minute frequency of the stocks in the Russell3000 index. The time series of the considered stocks can be found in \textsf{time\_series/pr\_07\_*.csv}.
\end{itemize}

\section{Analysis}

\begin{itemize}
\item In figure \ref{fig_ge} we depict a snapshot of the metaorders executed on GE.

\item In order to quantify the heterogeneity of the size of the metaorders contained in the dataset, we produce the histogram of the daily rate $Q/V_D$, figure \ref{fig_stat}. Each bin contains the same number of metaorders, i.e. the bins are produced by means of the function \textsf{py.percentile}. The plot of the estimated probability density function is represented on log-log in order to highlight the wide range of the distribution.

\item We measure the temporary price impact as a function of the daily rate $\pi = Q/V_D$, figure \ref{fig_imp_tmp}. We bin the data in 30 evenly-populated bins. For each bin we calculate the average daily rate $\pi$ and the average temporary impact by means of a \textsf{groupby} function. We then perform a fitting of the parameters considering a power-law, $f(x)= Yx^{\delta}$, and a logarithmic, $g(x) = a \log_{10}(1+bx)$,  function, by means of \textsf{optimize.curve\_fit}. Due to the fact that we perform a non-linear fit, we sample several initial parameters and we select the ones with maximum likelihood, in order to avoid to be stuck on a local minima of the minimisation algorithm. Unfortunately the size of the considered dataset is too small to obtain a clear signal of the impact.

\item We measure the transient price impact for metaorders with a duration $D$ (in minutes) such that $10 \le D \le 400$ and a participation rate $\eta$ such that $0.01 \le \eta \le 0.3$. For each metaorder we extract the time series of the price during the execution from the time series in the public dataset, we measure how the price impact built up during the execution and we make an average of all the time series. Each metaorder contribute to the average until the end of the execution. The results are presented in figure \ref{fig_imp_tra}, we observe that the price impact is a concave function of time. 

\end{itemize}

\begin{figure}[t] 
  \centering
 	\includegraphics[width=1\textwidth]{../plot/ge.pdf}
  \caption{Time series of metaorders active on the market for GE in the period January-February 2007. Buy (Sell) metaorders are depicted in blue (red). The thickness of the line is proportional to the metaorder participation rate. More metaorders in the same instant of time give rise to darker colours. Each horizontal line is a trading day. We observe very few blanks, meaning that there is almost always an active metaorder from our database, which is of course only a subset of the number of orders that are active in the market.}
  \label{fig_ge}
\end{figure}

\begin{figure}[t] 
  \centering
 	\includegraphics[width=1\textwidth]{../plot/stat_pi.pdf}
  \caption{Estimation of the probability density function of the daily fraction $\pi$. The plot is in log-log scale.}
  \label{fig_stat}
\end{figure}

\begin{figure}[t] 
  \centering
 	\includegraphics[width=1\textwidth]{../plot/imp_1d.pdf}
  \caption{Measured temporary market impact $\mathcal{I}_{tmp}(\pi )$ of a metaorder as a function of the daily rate $\pi$, defined as the ratio of the traded volume and the daily volume. The scale is double logarithmic; the dashed read line is the best fit to a power-law and the solid blue curve is the best fit to a logarithm.}
  \label{fig_imp_tmp}
\end{figure}


\begin{figure}[t] 
  \centering
 	\includegraphics[width=1\textwidth]{../plot/imp_tra.pdf}
  \caption{Measured transient price impact $\mathcal{I}(\pi )$ of a metaorder as a function of time $t$. }
  \label{fig_imp_tra}
\end{figure}














\end{document}